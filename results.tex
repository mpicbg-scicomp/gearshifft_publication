Based on the experiences made for \cite{preibisch2014efficient, schmid2015real}, this section will discuss results obtained with \gearshifft{} on various hardware in order to showcase the capabilities of \gearshifft{}. We will assume the motivation of a developer seeking to optimize the use of FFTs in the context of the aforementioned publications, i.e. 3D real-to-complex transforms with contiguous single-precision input data. If not stated otherwise, this is the transform type assumed for all illustrations hereafter. 

Expeditions into other use cases will be made where appropriate. The curious reader may rest assured that a more comprehensive study is easily possible with \gearshifft{}, however the mere multiplicity of all possible combinations and use cases of FFT render it neither feasible nor practical to discuss 1D or 2D in a comprehensive fashion as well.

For this study, we will concentrate on three modern and current FFT implementations available free of charge: FFTW ($3.3.5$, on x86 CPUs), cuFFT ($8.0.44$, on nVidia GPUs) and clFFT ($2.12.2$, on x86 CPUs or nVidia GPUs). We consider this the natural starting point of developers beyond possible domain specific implementations. It should be noted, that this will infer not only a study in terms of hardware performance, but also how well the APIs designed by the authors of FFTW, clFFT and cuFFT are documented, understood and used in practice. We consider both hardware and cognitive/developer performance a virtue of equal importance.

\subsection{Experimental Environment}
\label{ssec:env}

The results presented in the following sections were collected on three systems:

\begin{table}[htp]
  \centering
  \caption{Benchmark Hardware}
  \label{tab:hardware}
  \begin{tabular}{l|llll}
    \toprule
               & \multicolumn{2}{c}{\textbf{Taurus}}           & \textbf{Hypnos}           & \textbf{Islay}                            \\
               & \multicolumn{2}{c}{HPC Cluster \cite{taurus}} & HPC Cluster \cite{hypnos} & Workstation                               \\
               & K80 node                                      & K20Xm node                &                     &                     \\
    \midrule
    CPU vendor & Intel                                         & Intel                     & Intel               & Intel               \\
    CPU family & Haswell Xeon                                  & Sandybridge Xeon          & Haswell Xeon        & Haswell Xeon        \\
    CPU model  & E5-2680 v3                                    & E5-2450                   & E5-2603 v3          & E5-2640 v3          \\
    sockets    & $2$                                           & $2$                       & $2$                 & $2$                 \\
    RAM        & \SI{64}{\gibi\byte}                           & \SI{48}{\gibi\byte}       & \SI{64}{\gibi\byte} & \SI{64}{\gibi\byte} \\
    \midrule
    GPU vendor & NVIDIA                                        & NVIDIA                    & NVIDIA              & NVIDIA              \\
    GPU family & Tesla                                         & Tesla                     & Tesla               & GeForce             \\
    GPU        & 4x K80                                        & 2x K20x                   & 1x P100             & 1x GTX 1080         \\
    GPU memory & 8x \SI{12}{\gibi\byte}                        & \SI{6}{\gibi\byte}        & \SI{16}{\gibi\byte} & \SI{8}{\gibi\byte}  \\
    GPU driver & $367.48$                                      & $367.48$                  & $367.48$            & $367.57$            \\
    \midrule
    OS         & RHEL $7.2$                                    & RHEL $7.2$                & Ubuntu $14.04.3$    & CentOS $7.2$        \\
               & 64-bit                                        & 64-bit                    & 64-bit              & 64-bit              \\
    \bottomrule
  \end{tabular}
\end{table}

% \begin{itemize}
% \item \emph{Taurus HPC cluster}\cite{taurus} running RHEL 7.2
%   \begin{itemize}
%   \item \emph{K80 node}: 2x Intel(R) Xeon(R) CPU E5-2680 v3 (12 cores) @ 2.50GHz, 64 GB RAM, 4x NVIDIA Tesla K80 (12 GB GDDR5 RAM) GPUs 
%   \item \emph{K20X node}: 2x Intel(R) Xeon(R) CPU E5-2450 (8 cores) @ 2.10GHz, 48 GB RAM, 2x NVIDIA Tesla K20x (6 GB GDDR RAM) GPUs 
%   \end{itemize}
% \item \emph{Hypnos HPC cluster}\cite{hypnos} running Ubuntu 14.04.3:\newline
%   2x Intel(R) Xeon(R) CPU E5-2603 v3 @ 1.60GHz, 64 GB RAM (2.67 GB per core), 1x NVIDIA Tesla P100 (16 GB HBM2 RAM) GPUs via PCIe 
% \item  \emph{Dell workstation} running CentOS 7.2:\newline 
%   2x Intel(R) Xeon(R) CPU E5-2640 v3 @ 2.60GHz, 64 GB RAM, 1x NVIDIA GeForce GTX 1080 (8 GB GDDR5X RAM)
% \end{itemize}

All systems presented in \cref{tab:hardware} will be used for the benchmarks in this section. Access was performed via a \texttt{ssh} session without running a graphical user interface running on the target system. All measurements used the GNU compiler collection (GCC, \cite{stallman2001using}) version 5.3.0 as the underlying compiler if not stated otherwise. All used GPU implementations on nVidia hardware interfaced with the proprietary driver listed in \cref{tab:hardware} and used the infrastructure provided by CUDA $8.0.44$ if not stated otherwise. 

In order to generate one data set, a set arrays of arbitrary shapes is provided to a specific FFT API. The configuration files thereof can be accessed via \cite{gearshifft_github}. The shape configurations are separated in groups: \texttt{powerof2} (all dimensions are powers of $2$), \texttt{radix357} (all dimensions are either powers of $3$, $5$ or $7$) and \texttt{oddshape} (all dimensions are powers of $19$ in order to emulated very uncommon signal sizes). These configurations were generated in order to probe the FFT implementations for a wide spectrum of possible applications.  

The FFT calls to benchmark are executed five times each. From this, the arithmetic mean and sample standard deviations are used for figures presented below. As the number of repetitions is a configurable parameter of \gearshifft{}, we leave it to the user to produce a more comprehensive data set than used for this publication. We consider five repetitions enough at this point to show and discuss several aspects of performance and usability of \gearshifft{} and the FFT libraries under study.  

%TODO: why maximum size of transforms?

\subsection{Time To Solution}
\label{ssec:tts}

We begin the discussion with the classical use case for developers that might be accustomed to small size transforms. As such, an out-of-place transform with \texttt{powerof2} signal shapes will be assumed. The memory volume required for this operation amounts to the real input array plus an equally shaped complex output array.   

\begin{figure}[!tbp]
  \centering
  \includegraphics[width=\textwidth]{figures/results_tts_legend.pdf}
  \subfloat[linear scale]{\includegraphics[width=0.45\textwidth]{figures/results_tts_a.pdf}\label{fig:tts_a}}
  \hfill
  \subfloat[log10-log2 scale]{\includegraphics[width=0.45\textwidth]{figures/results_tts_b.pdf}\label{fig:tts_b}}
  \caption{Time-to-solution for power-of-2 3D single-precision real-to-complex forward transforms using FFTW (\texttt{FFTW\_ESTIMATE}) and cuFFT. \cref{fig:tts_b} shows the same data as \cref{fig:tts_a} but in a log10-log2 scale.}
  \label{fig:tts}
\end{figure}

\cref{fig:tts} reports a comparison of runtime results of power-of-2 single-precision real-to-complex forward transforms from FFTW and cuFFT. It is evident that given the largest device memory available of  \SI{16}{\gibi\byte}, the GPU data does not yield any points higher than \SI{8}{\gibi\byte}. Note that the total time reflects the time to set up a plan, allocate memory on device, perform the data transfer onto the device, execute the FFT, transfer the result back to the host and clean up the used plan and the allocated memory. \cref{fig:tts_a} shows that the oldest GPU generation used in this comparison yields the slowest results for input signals in the order of \SIrange{1}{2}{\gibi\byte}. All other and more recent GPU models supersede FFTW using all available cores in this node. Any judgment on the superiority of cuFFT over FFTW can be considered premature at this point, as FFTW was used with the \texttt{FFTW\_ESTIMATE} planner flag.

\begin{figure}[!tbp]
  \centering
  \includegraphics[width=\textwidth]{figures/results_plan_flags_legend.pdf}
  \subfloat[time to solution]{\includegraphics[width=0.45\textwidth]{figures/results_plan_flags_a.pdf}\label{fig:plan_flags_a}}
  \hfill
  \subfloat[time for forward transform only]{\includegraphics[width=0.45\textwidth]{figures/results_plan_flags_b.pdf}\label{fig:plan_flags_b}}
  \caption{Time-to-solution for power-of-2 3D single-precision real-to-complex forward transforms using FFTW (\texttt{FFTW\_ESTIMATE}) and cuFFT. \cref{fig:plan_flags_a} report the complete time to solution, whereas \cref{fig:plan_flags_b} is limited to the time spent for the execution of the forward transform only. Both figures use a in a log10-log2 scale.}
  \label{fig:fftw_plan_flags}
\end{figure}

\cref{fig:fftw_plan_flags} compares the time-to-solution to the actual time spent for the FFT operation itself. This illustration makes the cost and the benefit of higher planning flags than \texttt{FFTW\_ESTIMATE} obvious. Where \texttt{FFTW\_MEASURE} imposes a runtime penalty of 1 to 2 orders of magnitude with respect to \texttt{FFTW\_ESTIMATE}, it offers superior performance. The careful observer has noticed that the planning times for \texttt{FFTW\_MEASURE} become prohibitively large for large input data sizes and reach minutes for data sets in the Gigabyte range. This is a well-known feature of FFTW as the authors note in \cite{FFTW05}:

\begin{center}
  ``In performance critical applications, many transforms of the same
  size are typically required, and therefore a large one-time cost is
  usually acceptable.''
\end{center}
 
\gearshifft{} allows to dissect this problem further and isolate the planning time only.

\begin{figure}[!tbp]
  \centering
  \subfloat[Fig A.]{\includegraphics[width=0.45\textwidth]{figures/results_plan_time_a.pdf}\label{fig:plan_time_a}}
  \hfill
  \subfloat[Fig B.]{\includegraphics[width=0.45\textwidth]{figures/results_plan_time_b.pdf}\label{fig:plan_time_b}}
  \caption{Time-to-plan for power-of-2 single-precision real-to-complex forward transforms using FFTW, cuFFT and clFFT. \cref{fig:plan_time_a} reports the complete time to plan for 3D transforms, whereas \cref{fig:plan_time_b} is limited to 1D transforms. Both figures use a log10-log2 scale.}
  \label{fig:plan_time}
\end{figure}

\cref{fig:plan_time} illustrates the problem to it's full extent. \texttt{FFTW\_MEASURE} consumes up to 3-4 orders of magnitude more time to produce a plan than a standard (GPU based libraries) or \texttt{FFTW\_ESTIMATE} based planner call especially for large input shapes, see \cref{fig:plan_time_a}. We observed that FFTW wisdom cannot be generated for 2D or 3D out-of-place transforms. Therefor, we compare the 3D planning with it's counterpart in 1D in \cref{fig:plan_time_b}. At input sizes of \SI{128}{\mebi\byte} in 1D, the planning phase exceeds the duration of \SI{100}{\second}. In practice, this imposes a challenge on the client to the FFTW API. Not only is the time to solution affected by this behavior which is a crucial quantity if FFT-heavy applications are run in an HPC environment. Here the runtime of applications needs to be know to some extent in order to allow efficient and rapid job placement in a cluster. Further, the developer interfacing with FFTW has to create infrastructure (thread-safe static singleton objects or similar come to mind) in order to perform the planning of FFTW only once and reuse the resulting plan as much as possible.

\subsection{Comparing CPU versus GPU runtimes}
\label{ssec:cpu_vs_gpu}

The last section finished by discussing a design artifact, that the FFTW authors introduced in their API and which the other FFT libraries discussed here adopted. Another important question typically asked is if GPU accelerated FFT implementations are really faster than their CPU equivalents. Although this question cannot be answered comprehensively in our study, we would like to point out several aspects of it. First of all, modern GPU are connected via the PCIe bus to the host system in order to transfer data, receive instructions and to be supplied with power. This imposes a severe bottleneck to data transfer and is sometimes neglected during library design. Therefor, the time for data transfer needs to be accounted for be removed it from the measurement. \gearshifft{}s results data model offers access to each individual step of a transformation. With it, it is possible to isolate the time for the FFT transform only.

\begin{figure}[!tbp]
  \centering
  \includegraphics[width=\textwidth]{figures/results_r2c_fwd_legend.pdf}
  \subfloat[3D transforms]{\includegraphics[width=0.45\textwidth]{figures/results_r2c_fwd_a.pdf}\label{fig:r2c_fwd_a}}
  \hfill
  \subfloat[1D transforms]{\includegraphics[width=0.45\textwidth]{figures/results_r2c_fwd_b.pdf}\label{fig:r2c_fwd_b}}
  \caption{Time for computing power-of-2 single-precision real-to-complex forward transforms using the 3D API \cref{fig:r2c_fwd_a} and the 1D API clFFT \cref{fig:r2c_fwd_b}. Both figures use a log10-versus-log2 scale.}
  \label{fig:r2c_fwd}
\end{figure}

\cref{fig:r2c_fwd} shows the runtime spent for computing only the forward FFT for real single precision input data. This illustration is a direct measure for the quality of the implmentation and the hardware underneath. For the 3D case in \cref{fig:r2c_fwd_a}, we see that at the time of writing, FFTW on a double socket Haswell Intel Xeon E5 CPU provides very compelling performance if the input data is not larger than \SI{1}{\mebi\byte}. Above this limit, the GPU implementations offer a clear advantage by up to one order of magnitude. The current Pascal generation GPUs used with cuFFT provide the best performance, which does not come by surprise as both cards are equipped with GDDR5X or HBM2 memory which are clearly beneficial for an operation that yields low computational complexity such as the FFT. In the 1D case of \cref{fig:r2c_fwd_b}, the same observations must be made with even more certainty. The cross-over of FFTW and the GPU libraries occurs at an earlier point of \SI{64}{\kibi\byte}.  

\subsection{Non-power-of-2 transforms}
\label{ssec:nonpowerof2}

As power-of-2 shaped input signals do naturally fit the memory hierarchy found on modern CPUs and GPUs, the transformation of non-power-of-2 input signals could be considered a computational challenge. In a lot of conversations, the hypothesis is still communicated very often, that input signals should be padded to power-of-2 shapes in order to achieve the highest possible performance. As discussed in \cref{sec:motivation}, we evaluated the runtimes of input signals that have dimensions which are powers of $3,5,7$ (referred to as \texttt{radix357}) and to signals with a dimension being a power of 19 (referred to as \texttt{oddshape}) and which are powers of $2$ (referred to as \texttt{powerof2}) to probe the availability and quality of the mathematical typically used to perform FFTs on these. 

\begin{figure}[!tbp]
  \centering
  \includegraphics[width=\textwidth]{figures/results_non_power_of_2_legend.pdf}
  \subfloat[Time for FFT]{\includegraphics[width=0.45\textwidth]{figures/results_non_power_of_2_a.pdf}\label{fig:non_power_of_2_a}}
  \hfill
  \subfloat[Time to Plan]{\includegraphics[width=0.45\textwidth]{figures/results_non_power_of_2_b.pdf}\label{fig:non_power_of_2_b}}
  \caption{Time for computing single-precision real-to-complex forward transforms using the 3D API \cref{fig:non_power_of_2_a} and the time required to plan the aforementioned execution \cref{fig:non_power_of_2_b}. Both figures use a log10-versus-log2 scale.}
  \label{fig:non_power_of_2}
\end{figure}

Even though our findings in \cref{fig:non_power_of_2} are restricted to FFTW and cuFFT for the sake of simplicity, the figure shows that this urban myth of padding for a power-of-2 to achieve better performance can be falsified to some extent. Clearly, power-of-2 transforms are the fastest among the three categories we chose. \cref{fig:non_power_of_2_a} illustrates that FFTW is able to offer almost identical performance for different input signal shape types across a very wide range of input signal volumes. This however comes at the expense of long planning times, see \cref{fig:non_power_of_2_b}. The situation is not so clear for cuFFT. For large input signals, a runtime difference of up to one order of magnitude on the K80 and the P100 can be observed seen. For small input signal sizes, the performance of the K80 is even superior to the P100. Further, we can observe that the cuFFT runtimes varies for different input signal shapes especially for the case of large input signal arrays. We can only hypothesize about the root cause of this behavior whether it is due to lack of implemented mathematical approaches or exploitation of hardware features. Thus, a padding to power-of-2 when using cuFFT might be justified if enough memory is available on the device.     

\subsection{Data Types}
\label{ssec:data_types}

In this section we would like to address r2c versus c2c!

\subsection{Memory Modes}
\label{ssec:mem_mode}

In this section we would like to address inplace versus out-of-place transforms!
