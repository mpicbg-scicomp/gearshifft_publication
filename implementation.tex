
% - [open source] framework: c++14 with help of boost
% - demonstrate usage of gearshifft framework
% - cufft, clfft, fftw
% - issues?
% - #runs, cmake, project structure
% - libraries : mem checks

\gearshifft{} is developed as an open-source framework using C++14 standard and Boost Unit Test Framework for managing the tree of benchmark cases.
The frontend API is basically a wrapper for the FFT client operations like plan initialisation, forward or backward transformation as explained in \cref{sec:benchmark_model}.
The interface for the FFT client is designed to integrate any given FFT library, that provides forward and backward Fourier transforms.
The wrapper code leverages C++ templates and compile-time constant expressions.
This yields minimal overhead at runtime and provides a type-agnostic interface, i.\,e. it is not fixed to single or double precision.
However, the benchmark runtime environment allocates memory for samples and performance data.

The frontend interface requires the user to implement the context class and a class with methods for the FFT client routines.
The context class in \cref{lst:implcontext} is instantiated only once for the application lifetime.
The FFT client implementation class in \cref{lst:implfft} is instantiated once per benchmark run and follows the ``resource allocation is initialisation'' (RAII) pattern.


\begin{lstlisting}[caption={Context class required by gearshifft frontend API},label={lst:implcontext}]
struct Context {
  /// title for all benchmarks associated to this context
  static std::string title();
  /// list all compute devices
  static std::string get_device_list();
  /// information for current device
  std::string get_used_device_properties();
  /// creates context
  void create();
  /// destroys context
  void destroy();
};
\end{lstlisting}
\begin{lstlisting}[caption={FFT client implementation required by gearshifft frontend API},label={lst:implfft}]
template<
 typename TFFT, // e.g. gearshifft::FFT_Inplace_Real, ...
 typename TPrecision, // e.g. double, float, ...
 size_t   NDim        // 1,..,3
>
struct FFT_Client_Impl {
  FFT_Client_Impl();
  ~FFT_Client_Impl();
  void allocate();
  size_t getAllocSize();
  size_t getTransferSize();
  size_t getPlanSize();
  void init_forward();
  void init_inverse();
  void execute_forward();
  void execute_inverse();
  template<typename THostData>
  void upload(THostData* input);
  template<typename THostData>
  void download(THostData* output);
  void destroy();
};
\end{lstlisting}

\begin{lstlisting}[caption={FFT wrapper class},label={lst:fftabstract}]
template<
 typename T_FFT, // e.g. FFT_Inplace_Real ..
 typename T_ReusePlan, // can plan be reused ?
 template <typename,typename,size_t,typename... > typename T_Client,
 typename T_DeviceTimer,
 typename... T_ClientArgs
>
struct FFT : public T_FFT {
  template<typename T_Result, typename T_Vector, size_t NDim>
  void operator(/*..*/);
};
\end{lstlisting}

The wrapper for the FFT client and the layout of measurement is implemented in the \mc{gearshifft::FFT} functor.
Its template parameters are shown in \cref{lst:fftabstract}. 
\mc{T_FFT} is one of \gearshifft{} predefined types \mc{FFT_Inplace_Real}, \mc{FFT_Outplace_Real}, \mc{FFT_Inplace_Complex} and \mc{FFT_Outplace_Complex}.
Most libraries allow plan reusage, even fftw when \mc{FFTW_MEASURE} is not used.
Reusing a plan means to use only one plan handle. After forward transform has been finished the same plan handle is used for backward transform.
This might save memory as there are no two plans allocated at the same time.
If it is not possible to reuse the plan handle, then plan handles for forward and inverse transform are allocated in the initialisation phase.
\mc{T_Client} defines the FFT client class like \mc{CuFFTImpl} or \mc{FftwImpl}.
The template parameters are at least the \mc{T_FFT} type (\mc{FFT_Inplace_Real}, \ldots), the precision type (float, \ldots) and the FFT dimension (1, 2 or 3), see also \cref{lst:implfft}. If the client needs more template arguments the variadic template \mc{T_ClientArgs} is given to the \mc{T_Client} instantiation as fourth parameter (expansion of a parameter pack).
The \mc{T_DeviceTimer} allows device located time measurements and is only implemented for CUDA/cuFFT library at the moment.

\begin{lstlisting}[caption={Define FFT client types for corresponding FFTs},label={lst:implfftusing}]
using Inplace_Real = gearshifft::FFT<
 gearshifft::FFT_Inplace_Real, FFT_Client_Impl, TimerCPU >;
\end{lstlisting}

Finally, the FFT client and the FFT wrapping functor are combined in \cref{lst:implfftusing} to obtain the type at compile-time, which is invoked in benchmark.cpp as \cref{lst:implfftusingp2} illustrates. The \mc{gearshifft::List} is a compile-time constant list, which holds the different template instantiations of the FFT client. 

\begin{lstlisting}[caption={Using FFT client types to run the benchmarks},label={lst:implfftusingp2}]
using namespace gearshifft;
using Context           = MyFFTClient::Context;         
/// meta list can be extended with further types
using FFTs              = List<MyFFTClient::Inplace_Real>;
using Precisions        = List<float, double>;   
/// if iFFT(FFT(x)) must not be divided by number of elements
using FFT_Is_Normalized = std::false_type;       

int main( int argc, char* argv[] )                       
{                                                        
  try {                                                  
    Benchmark<Context> benchmark;
    benchmark.configure(argc, argv);                     
    benchmark.run<FFT_Is_Normalized, FFTs, Precisions>();
  } catch(const std::runtime_error& e) {
    std::cerr << e.what() << std::endl;                  
    return 1;                                            
  }                                                      
  return 0;                                              
}                                                        
\end{lstlisting}


Now after the frontend API has been introduced, the backend of \gearshifft{} is discussed.
A benchmark is defined to collect performance indicators of a set of operations, and repeats the execution several times to obtain reliable statistics. Different parameters like precision, FFT extents, transform variant, device type or FFT library relate to different benchmarks.
\gearshifft{} controls many of them by command line arguments. The FFT libraries are related to different \gearshifft{} binaries (\mc{gearshifft_cufft}, \ldots).
The backend of \gearshifft{} uses Boost Unit Test Framework to generate the benchmark instances. % @todo ref

The measurement layout and benchmark framework is illustrated in \cref{fig:framework}.
One single run comprises time measurement of each operation (allocate, \ldots). 
%allocate, init forward, upload, execute forward, init inverse, execute inverse, download and destroy
The total time measures from \mc{allocate} over forward and inverse transforms to \mc{destroy}.
Furthermore, the memory footprint of the FFT client is recorded to answer the question, which FFT library consumes minimal memory.
%i.\,e. the FFT client \mc{T_Client} is intantiated to run its operations for plan generation, FFTs and data management.

The functor \mc{FFT} contains the FFT client operations, wrapped with time measurements, and is invoked multiple times (compile-time constant in application.hpp). The input data buffer is hold by \mc{BenchmarkExecutor} and a copy is given to the \mc{FFT} functor.


% implies plan reuse
\begin{figure}
\centering
%align=center,rounded corners,inner sep=5pt,rectangle,draw,
\tikzset{class/.style={inner sep=5pt,font=\footnotesize}}
\newcommand{\pclass}[5][]{
\ifthenelse { \equal {#1} {} }
 {\node[class] (#5) at (#3,#4) {#2};}
 {\node[class] (#5) at (#3,#4) {%
\begin{tabular}{c}\scriptsize{<<#1>>}\\#2\end{tabular}%
};}
}
\begin{tikzpicture}
\tikzset{gr1/.style={fill=black!15}}
\tikzset{bts/.style={draw,circle,inner sep=2pt}}
\tikzset{btc/.style={draw,circle,inner sep=2pt,fill=black}}
%
\begin{scope}[yshift=3.5cm,xshift=-2.9cm]
\node[bts] (b0) at (0,0) {};
\node[bts] (b10) at (-0.5,-0.6) {}; \draw (b10) -- (b0);
\node[bts] (b11) at (0.5,-0.6) {}; \draw (b11) -- (b0);
\node[btc] (b20) at (-0.75,-1.3) {}; \draw (b20) -- (b10);
\node[btc] (b21) at (-0.25,-1.3) {}; \draw (b21) -- (b10);
\node[btc] (b22) at ( 0.25,-1.3) {}; \draw (b22) -- (b11);
\node[btc] (b23) at ( 0.75,-1.3) {}; \draw (b23) -- (b11);
\node[font=\scriptsize] at (0, 0.3) {Boost Test Suites};
\node[font=\scriptsize] at (0,-1.7) {Boost Test Cases};
\end{scope}
% 
\begin{scope}[xshift=2cm]
\begin{scope}
\pclass{Benchmark}{-2}{4}{b}
\pclass[Functor]{BenchmarkSuite}{-2}{3.2}{bs}
\pclass[Functor]{BenchmarkExecutor}{-2}{2.1}{be}
\pclass[Functor]{FFT}{-2}{1.0}{fft}
\end{scope}
\begin{scope}
\pclass[Singleton]{Application}{1.5}{4}{app}
\pclass[Realisation]{Context}{1.5}{2.5}{ctx}
\pclass[Realisation]{FFTClient}{1.5}{1.1}{impl}
\end{scope}
\end{scope}
\matrix[
 minimum height=1.5em,
 matrix of nodes,
 row sep=-\pgflinewidth,
 column sep=-\pgflinewidth,
 text depth=2.5ex,
 text height=1.5ex,
 text width=3.6em,
 align=center,
 nodes in empty cells,
 row 1/.style={nodes={rectangle,draw,minimum width=3em,font=\scriptsize}}
]
(mf) at (0,0) {
allocate &
init\linebreak forward &
|[gr1]| upload &
|[gr1]| execute\linebreak forward &
init\linebreak inverse &
|[gr1]| execute\linebreak inverse &
|[gr1]| download &
destroy\\
};
\draw (mf-1-1.north west) ++(-0.75em,0.5em) coordinate (ctl) -- ([xshift=0.75em,yshift=0.5em]mf-1-8.north east) coordinate (cr);
\draw[dotted] (ctl) -- ++(-2ex,0); \draw[dotted] (cr) -- ++(2ex,0);
\draw (mf-1-1.south west) ++(-0.75em,-0.5em) coordinate (cl) -- ([xshift=0.75em,yshift=-0.5em]mf-1-8.south east) coordinate (cr);
\draw[dotted] (cl) -- ++(-2ex,0); \draw[dotted] (cr) -- ++(2ex,0);
% total time
\draw[thick,dashed,|-|] (mf-1-1.south west) ++(0,-1.5em) -- ([yshift=-1.5em]mf-1-8.south east) node[pos=0.5,fill=white,font=\small\itshape] {total time};
% (mf-1-1.south west) -- ++(0,-1.5em) -| (mf-1-8.south east) node[pos=0.25,fill=white,font=\small] {total time};

% \draw[-latex] (b) -- (bs);
% \draw[-latex] (fft) -- (ctl-|fft);
\draw[black!50] (b.south west) -- (b.south east);
\draw[black!50] (bs.south west) -- (bs.south east);
\draw[black!50, dashed] (bs.south west) -- ++(-8em,0); % test suite marker
\draw[black!50] (be.south west) -- (be.south east);

\draw[densely dashed,-angle 60] (app) -- (ctx);
\draw[-angle 60] (b) -- (app.west|-b);
% \draw[densely dashed,-angle 90] (be.east) -| (impl.north);
% \draw[densely dashed,-open triangle 60] (impl) -- (fft.east|-impl) node[midway] (q) {};
\draw[densely dashed,-angle 90] (fft) -- (impl.west|-fft) node[midway] (q) {};
\draw[densely dashed] (q) -- (ctl-|q);
\end{tikzpicture}
 \caption{The benchmark framework of \gearshifft{} using Boost UTF and a realised FFT interface; Here only FFT interfaces are shown, that are measured (gray operations are measured by device timers if required); Context also has an implicit interface, which is omitted here.}
 \label{fig:framework}
\end{figure}


