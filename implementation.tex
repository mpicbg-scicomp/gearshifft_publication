
% - [open source] framework: c++14 with help of boost
% - demonstrate usage of gearshifft framework
% - cufft, clfft, fftw
% - issues?
% - #runs, cmake, project structure
% - libraries : mem checks

\gearshifft{} is developed as an open-source framework using C++14 standard and Boost Unit Test Framework for managing the benchmark trees. Before discussing the design of \gearshifft{} in detail, it is instructive to provide a brief introduction to state of the FFT libraries that will be subject to the study in \cref{sec:results},

\subsection{Using a Modern FFT Library}
\label{ssec:modern_ffts}

The design of many FFT libraries is based on the architecture and API of FFTW $3.0$. Beyond memory management to allocate and fill the input signals and prepare the output signal arrays if needed, there are two interactions that any client to FFTW has to perform in order to execute an FFT on the aforementioned input signal. First a plan data structure has to be created and filled using a planner. For this, the FFT problem is defined in terms of rank (1D, 2D or 3D), shape of the input signal, type of the input signal (single or double precision of real or complex inputs), type of the transformation (real-to-complex, complex-to-complex, real-to-real) and memory mode of the transformation (inplace versus out-of-place). The FFT problem description is then used as input parameters to the planner. The planner is a piece of code inside FFTW that henceforth tries to find the best suited radix factorization based on the shape of the input signal. It then performs several FFTs on the input data to sample the runtime of different FFT implementations available inside FFTW. The sample of runtimes is then used to find the optimal implementation of the FFT. After the plan has been created, it is used to execute the FFT itself.

\begin{lstlisting}[caption={Minimal usage example of the FFTW single precision real-to-complex planner API. Memory management is omitted.},label={lst:fftw_example}]
int shape[] = {32,32,32};
fftw_plan r2c_plan = fftw_plan_dft_r2c(
 3,                       //rank, here 3D 
 shape,                   //shape of the input
 (float *) input_buffer,  //input data array
 (fftwf_complex *) output,//output data array
 FFTW_ESTIMATE            //planner flag
 );
fftwf_execute(r2c_plan);
\end{lstlisting}

\cref{lst:fftw_example} illustrates the FFTW API for a single precision real-to-complex out-of-place transform. It is important to note that FFTW offers the user the freedom to choose the degree of optimization for finding the most optimal FFT implementation for the signal at hand. \cref{lst:fftw_example} uses the \texttt{FFTW\_ESTIMATE} flag, which is described by the FFTW manual \cite{fftw_manual} as:
%
\begin{quote}
  ``\texttt{FFTW\_ESTIMATE} specifies that, instead of actual measurements of different algorithms, a simple heuristic is used to pick a (probably sub-optimal) plan quickly. With this flag, the input/output arrays are not overwritten during planning.''
\end{quote}
%
Fftw offers five levels for this planning flag in total. As covering all of them in this study is impractical, we selected two more: \texttt{FFTW\_MEASURE} and \texttt{FFTW\_WISDOM\_ONLY}. The FFTW manual describes them as:
%
\begin{quote}
  ``\texttt{FFTW\_MEASURE} tells FFTW to find an optimized plan by actually computing several FFTs and measuring their execution time. Depending on your machine, this can take some time (often a few seconds). FFTW\_MEASURE is the default planning option.\newline
  \texttt{FFTW\_WISDOM\_ONLY} is a special planning mode in which the plan is only created if wisdom is available for the given problem, and otherwise a NULL plan is returned.''
\end{quote}
%
In FFTW terminology, wisdom is a persistent data structure that \texttt{fftw\_wisdom} binary in the FFTW install tree can generate. It is meant to allow users to generate the plan for specific input data shapes offline to their application. The stored 'wisdom' can then be read from disk through the FFTW wisdom API. 
%
\begin{lstlisting}[caption={Minimal usage example of the cuFFT single precision real-to-complex planner API. Memory management is omitted.},label={lst:cufft_example}]
int N = 32;
cufftHandle plan;
cufftPlan3d(&plan, N, N, N, CUFFT_R2C);
cufftExecC2C(plan, input_buffer, output, CUFFT_FORWARD);
\end{lstlisting}
%
As discussed above, the studied GPU based FFT implementations follow this design structure without the planning flags however. This becomes evident in \cref{lst:cufft_example}. 
 

\subsection{The Architecture of \gearshifft{}}
\label{ssec:modern_ffts}


The frontend API is basically just a wrapper mapping the use case explained in \cref{sec:motivation}.
This interface is designed to integrate any given FFT library, that provides forward and backward Fourier transforms.
The wrapper code leverages C++ templates and compile-time constant expressions.
This yields minimal overhead at runtime and provides a type-agnostic interface, i.\,e. it is not fixed to single or double precision.
However, corresponding timer calls and benchmark routines for collecting data are required at runtime, so measurements will have to be verified.

The frontend interface requires the user to implement the context class and a class with implementation of FFT routines.
The context class in \cref{lst:implcontext} is instantiated only once for the application lifetime.
FFT implementation class in \cref{lst:implfft} is instantiated once per benchmark run and follows the ``resource allocation is initialization'' (RAII) pattern.


\begin{lstlisting}[caption={Context class required by gearshifft API},label={lst:implcontext}]
struct Context {
  /// title for all benchmarks
  static std::string title();
  /// list all compute devices
  static std::string get_device_list();
  /// information for current device
  std::string get_used_device_properties();
  /// creates context
  void create();
  /// destroys context
  void destroy();
};
\end{lstlisting}
\begin{lstlisting}[caption={FFT class required by gearshifft API},label={lst:implfft}]
template<
 typename TFFT, // e.g. gearshifft::FFT_Inplace_Real, ...
 typename TPrecision, // e.g. double, float, ...
 size_t   NDim        // 1,..,3
>
struct MyImpl {
  MyImpl();
  ~MyImpl();
  void allocate();
  size_t getAllocSize();
  size_t getTransferSize();
  size_t getPlanSize();
  void init_forward();
  void init_inverse();
  void execute_forward();
  void execute_inverse();
  template<typename THostData>
  void upload(THostData* input);
  template<typename THostData>
  void download(THostData* output);
  void destroy();
};
\end{lstlisting}
\begin{lstlisting}[caption={Using FFT class},label={lst:implfftusing}]
using Inplace_Real =
 gearshifft::FFT<gearshifft::FFT_Inplace_Real, MyImpl, TimerCPU>;
\end{lstlisting}

The wrapper for the user FFT class and the layout of measurement is given in the \lstinline!gearshifft::FFT! class.
Its are shown in \cref{lst:fftabstract}. The measurement framework is illustrated in \cref{fig:timings}.
\begin{lstlisting}[caption={FFT wrapper class},label={lst:fftabstract}]
template<
 typename TFFT,
 template <typename,typename,size_t,typename... > typename TPlan,
 typename TDeviceTimer,
 typename... TPlanArgs // is redirected to TPlan
>
struct FFT : public TFFT {
  template<typename T_Result, typename T_Vector, size_t NDim>
  void operator(/*..*/);
};
\end{lstlisting}

% implies plan reuse
\begin{figure}\label{fig:timings}
\centering
%align=center,rounded corners,inner sep=5pt,rectangle,draw,
\tikzset{class/.style={inner sep=5pt,font=\footnotesize}}
\newcommand{\pclass}[5][]{
\ifthenelse { \equal {#1} {} }
 {\node[class] (#5) at (#3,#4) {#2};}
 {\node[class] (#5) at (#3,#4) {%
\begin{tabular}{c}\scriptsize{<<#1>>}\\#2\end{tabular}%
};}
}
\begin{tikzpicture}
\tikzset{gr1/.style={fill=black!15}}
\tikzset{bts/.style={draw,circle,inner sep=2pt}}
\tikzset{btc/.style={draw,circle,inner sep=2pt,fill=black}}
%
\begin{scope}[yshift=3.5cm,xshift=-2.9cm]
\node[bts] (b0) at (0,0) {};
\node[bts] (b10) at (-0.5,-0.6) {}; \draw (b10) -- (b0);
\node[bts] (b11) at (0.5,-0.6) {}; \draw (b11) -- (b0);
\node[btc] (b20) at (-0.75,-1.3) {}; \draw (b20) -- (b10);
\node[btc] (b21) at (-0.25,-1.3) {}; \draw (b21) -- (b10);
\node[btc] (b22) at ( 0.25,-1.3) {}; \draw (b22) -- (b11);
\node[btc] (b23) at ( 0.75,-1.3) {}; \draw (b23) -- (b11);
\node[font=\scriptsize] at (0, 0.3) {Boost Test Suites};
\node[font=\scriptsize] at (0,-1.7) {Boost Test Cases};
\end{scope}
% 
\begin{scope}[xshift=2cm]
\begin{scope}
\pclass{Benchmark}{-2}{4}{b}
\pclass[Functor]{BenchmarkSuite}{-2}{3.2}{bs}
\pclass[Functor]{BenchmarkExecutor}{-2}{2.1}{be}
\pclass[Functor]{FFT}{-2}{1.0}{fft}
\end{scope}
\begin{scope}
\pclass[Singleton]{Application}{1.5}{4}{app}
\pclass[Realisation]{Context}{1.5}{2.5}{ctx}
\pclass[Realisation]{FFTClient}{1.5}{1.1}{impl}
\end{scope}
\end{scope}
\matrix[
 minimum height=1.5em,
 matrix of nodes,
 row sep=-\pgflinewidth,
 column sep=-\pgflinewidth,
 text depth=2.5ex,
 text height=1.5ex,
 text width=3.6em,
 align=center,
 nodes in empty cells,
 row 1/.style={nodes={rectangle,draw,minimum width=3em,font=\scriptsize}}
]
(mf) at (0,0) {
allocate &
init\linebreak forward &
|[gr1]| upload &
|[gr1]| execute\linebreak forward &
init\linebreak inverse &
|[gr1]| execute\linebreak inverse &
|[gr1]| download &
destroy\\
};
\draw (mf-1-1.north west) ++(-0.75em,0.5em) coordinate (ctl) -- ([xshift=0.75em,yshift=0.5em]mf-1-8.north east) coordinate (cr);
\draw[dotted] (ctl) -- ++(-2ex,0); \draw[dotted] (cr) -- ++(2ex,0);
\draw (mf-1-1.south west) ++(-0.75em,-0.5em) coordinate (cl) -- ([xshift=0.75em,yshift=-0.5em]mf-1-8.south east) coordinate (cr);
\draw[dotted] (cl) -- ++(-2ex,0); \draw[dotted] (cr) -- ++(2ex,0);
% total time
\draw[thick,dashed,|-|] (mf-1-1.south west) ++(0,-1.5em) -- ([yshift=-1.5em]mf-1-8.south east) node[pos=0.5,fill=white,font=\small\itshape] {total time};
% (mf-1-1.south west) -- ++(0,-1.5em) -| (mf-1-8.south east) node[pos=0.25,fill=white,font=\small] {total time};

% \draw[-latex] (b) -- (bs);
% \draw[-latex] (fft) -- (ctl-|fft);
\draw[black!50] (b.south west) -- (b.south east);
\draw[black!50] (bs.south west) -- (bs.south east);
\draw[black!50, dashed] (bs.south west) -- ++(-8em,0); % test suite marker
\draw[black!50] (be.south west) -- (be.south east);

\draw[densely dashed,-angle 60] (app) -- (ctx);
\draw[-angle 60] (b) -- (app.west|-b);
% \draw[densely dashed,-angle 90] (be.east) -| (impl.north);
% \draw[densely dashed,-open triangle 60] (impl) -- (fft.east|-impl) node[midway] (q) {};
\draw[densely dashed,-angle 90] (fft) -- (impl.west|-fft) node[midway] (q) {};
\draw[densely dashed] (q) -- (ctl-|q);
\end{tikzpicture}
 \caption{The benchmark framework of \gearshifft{} using Boost UTF and a realised FFT interface; Here only FFT interfaces are shown, that are measured (gray operations are measured by device timers if required); Context also has an implicit interface, which is omitted here.}
\end{figure}


The backend of \gearshifft{} uses Boost Unit Test Framework to manage the benchmark instances.

