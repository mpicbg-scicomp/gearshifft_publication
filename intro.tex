Fast Fourier Transforms (FFTs, \citep{van1992computational}) are at the heart of many signal processing and phase space exploration algorithms. Examples for their substantial usage come from image reconstruction in life sciences \citep{preibisch2014efficient,schmid2015real}, amino acid sequence alignment in bioinformatics \citep{katoh2002mafft}, phase space reduction for weather simulations \citep{maronga2015parallelized}, option price analysis and prediction in financial mathematics \citep{hurd2010fourier} or machine learning \citep{dlstudy} to just name a few.

As input data grows in size with increasing experimental data production \citep{huisken2004optical} and simulation output bandwidths \citep{maronga2015parallelized}, input data to FFT libraries on the order of Gigabytes becomes the standard. With the advent of graphics processing units (GPUs) for scientific processing computing around the beginning of the 21st century and the subsequent availability of general purpose programming paradigms to program these \citep{du2012cuda}, vendor-specific and open-source libraries to perform FFTs on accelerators emerged (\cufft{} \citep{nvidia2010cufft} by Nvidia, open-source \clfft{} \citep{clfft}) to offer performance which supersedes traditional high-performance implementations running on standard Central Processing Units (CPUs) such as the open-source \fftw{} library \citep{FFTW05} or the Intel specific MKL \citep{intel2007intel}.

Also, the top ten sites listed of the fastest worldwide computer installations (Top500 \citep{meuer2011top500}) shows that the used hardware is by far not homogeneous in terms of vendor and composition. As this trend can be observed in practice even more, library architects and domain specialists are confronted with an essential question: Which FFT implementation works best on what hardware in terms of runtime or large signal sizes?  

To answer this question we propose an open-source benchmark package called \gearshifft{} \citep{gearshifft_github} that allows to benchmark available state-of-the-art FFT libraries in a reproducible, automated, comprehensive, open and vendor-independent fashion on CPUs and GPUs.

To our surprise, comprehensive and peer-reviewed benchmarks of FFT implementations across different hardware platforms have not been published extensively. Either only specific hardware is chosen for the benchmark \citep{park2015fast,eleftheriou2005performance,Akin:15} or only specific FFT implementation variants are tested \citep{shoc2010,dongarra2013hpc}. Aside of that, many performance benchmarks are tied to domain-specific implementations \citep{fialka2006fft} that either lack comprehensiveness or the ability to map the obtained results to other implementation requirements.

We thus conclude that a generic, comprehensive and open benchmark suite can help not only library authors and domain-specific developers to choose the best FFT library available. It will also support decision makers to choose the right technology if a FFT heavy workload is planned for a hardware installation. The discussion above motivates the following design goals of \gearshifft{}:

\begin{itemize}
\item open-source and free code
\item standardized output format for downstream statistical analysis
\item state-of-the-art build system
\item open and extensible architecture with generic interface
\item community-ready and vendor independent project infrastructure through version control and public accessibility
\end{itemize}

The remainder of this article is organized as follows: the motivation is laid out in section \ref{sec:motivation} followed by an introduction to modern FFT APIs and the discussion of the chosen implementation in section \ref{sec:implementation}. The largest part of the paper is dedicated to the discussion of first results in section \ref{sec:results} after which our conclusions are presented in section \ref{sec:summary}.

