Fast Fourier Transforms (FFTs, \citep{van1992computational}) are at the heart of many signal processing and phase space exploration algorithms. Examples for their substantial usage come from image reconstruction in life sciences \citep{preibisch2014efficient,schmid2015real}, amino acid sequence alignment in bioinformatics \citep{katoh2002mafft}, phase space reduction for weather forecasts \citep{maronga2015parallelized}, option price analysis and prediction in financial mathematics \citep{dempster2002spread,hurd2010fourier} to machine learning \citep{collobert2011torch7,jia2014caffe,abadi2016tensorflow} to just name a few.

As input data grows in size and variety both of experimental hardware \citep{huisken2004optical} and simulation outputs \citep{maronga2015parallelized}, input data on the order of Gigabytes to FFT libraries becomes the standard. With the advent of graphics processing units (GPUs) for scientific processing computing around the beginning of the 21st century and the subsequent availability of general purpose programming paradigms to program these \citep{du2012cuda}, vendor-specific and open-source libraries to perform FFTs on accelerators emerged (cuFFT \citep{nvidia2010cufft} by nVidia, open-source clFFT \citep{clfft}) to offer performance which supersedes traditional high-performance implementations running on standard Central Processing Units (CPUs) such as the open-source FFTW library \citep{FFTW97,FFTW05} or the Intel specific MKL \citep{intel2007intel}.

However, a review of the top ten sites listed of the fastest worldwide computer installations (Top500 \citep{meuer2011top500}) shows that the used hardware is by far not homogeneous. As this trend can be observed in practice even more, library architects and domain specialists are confronted with the question: which FFT implementation works best on what hardware in terms of runtime or large signal sizes?  
To answer this question we propose an open-source benchmark package called \gearshifft{} \citep{gearshifft_github} that allows to benchmark available state-of-the-art FFT libraries in a reproducible, automated, comprehensive, open and vendor-independent fashion on CPUs and GPUs.

To our surprise, comprehensive and open peer-reviewed benchmarks of FFT implementations across different hardware platforms have not been published extensively. Either only specific hardware is chosen for the benchmark \citep{park2015fast,eleftheriou2005performance} or only specific FFT implementation variants are tested \citep{shoc2010,dongarra2013hpc}. Aside of that many performance benchmarks are tied to domain-specific implementations \citep{fialka2006fft} that either lack comprehensiveness or the ability to map the obtained results to other implementation requirements.

Non-peer reviewed comparisons appear limited to CPU-only \citep{benchFFT}, CPU and ARM architectures \citep{roy_longbottom} or lack comprehensiveness by benchmarking only one single type of FFT algorithm \citep{fft_check}. Further, most of the listed non-peer reviewed benchmarks appear outdated and do not offer a community-ready contribution model for future results to mitigate the challenge to incorporate a large variety of hardware and software versions. Also, a lot of vendor-specific benchmarks exist. As their results are likely to be subject to marketing efforts and/or confirmation bias, they shall not be discussed here.

We thus conclude that a generic, comprehensive and open benchmark suite can help not only library authors and domain-specific developers to choose the best FFT library available. It will also help decision makers to choose the right technology if a FFT heavy workload is planned for a hardware installation about to be procured. The discussion above thus motivate the following design goals of \gearshifft{}:

\begin{itemize}
\item open-source and free code
\item standardized output format for downstream statistical analysis
\item state-of-the-art build system
\item open and extensible architecture with generic interface for libraries that have not been benchmarked yet
\item community-ready project infrastructure
\end{itemize}

The remainder of this article is organized as follows: the motivation is laid out in section \ref{sec:motivation} followed by an introduction to modern FFT APIs and the discussion of the chosen implementation in section \ref{sec:implementation}. The largest part of the paper is dedicated to the discussion of first results in section \ref{sec:results} after which our conclusions are presented in section \ref{sec:summary}.


%% CPU and related to accFFT library (not vendor-independent)
%% \citep{DBLP:journals/corr/GholamiHMB15}

% improvements of GPU implementations
% http://dl.acm.org/citation.cfm?id=1413373
%
% @article{sorensen2008accelerating,
%   title={Accelerating the nonequispaced fast Fourier transform on commodity graphics hardware},
%   author={S{\o}rensen, Thomas Sangild and Schaeffter, Tobias and Noe, Karsten {\O}stergaard and Hansen, Michael Schacht},
%   journal={IEEE Transactions on Medical Imaging},
%   volume={27},
%   number={4},
%   pages={538--547},
%   year={2008},
%   publisher={IEEE}
% }
